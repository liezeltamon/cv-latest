%-------------------------
% Resume in Latex
% Author : Sourabh Bajaj
% License : MIT
%------------------------

\documentclass[legalpaper,11pt]{article}

\usepackage{latexsym}
\usepackage[empty]{fullpage}
\usepackage{titlesec}
\usepackage{marvosym}
\usepackage[usenames,dvipsnames]{color}
\usepackage{verbatim}
\usepackage{enumitem}
%\usepackage[hidelinks]{hyperref}
\usepackage{hyperref}
\hypersetup{
    colorlinks=true,
    linkcolor=blue,
    filecolor=magenta,      
    urlcolor=blue%,
    %pdfpagemode=FullScreen
}

\usepackage{fancyhdr}
\usepackage[english]{babel}
\usepackage{tabularx}

\pagestyle{fancy}
\fancyhf{} % clear all header and footer fields
\fancyfoot{}
\renewcommand{\headrulewidth}{0pt}
\renewcommand{\footrulewidth}{0pt}

% Adjust margins
\addtolength{\oddsidemargin}{-0.5in}
\addtolength{\evensidemargin}{-0.5in}
\addtolength{\textwidth}{1in}
\addtolength{\topmargin}{-.5in}
\addtolength{\textheight}{1.0in}

\urlstyle{same}

\raggedbottom
\raggedright
\setlength{\tabcolsep}{0in}

% Sections formatting
\titleformat{\section}{
  \vspace{-4pt}\scshape\raggedright\large
}{}{0em}{}[\color{black}\titlerule \vspace{-5pt}]

%-------------------------
% Custom commands
\newcommand{\resumeItem}[2]{
  \item\small{
    \textbf{#1}{: #2 \vspace{-2pt}}
  }
}

\newcommand{\resumeSubheading}[4]{
  \vspace{-1pt}\item
    \begin{tabular*}{0.97\textwidth}[t]{l@{\extracolsep{\fill}}r}
      \textbf{#1} & #2 \\
      \textit{\small#3} & \textit{\small #4} \\
    \end{tabular*}\vspace{-5pt}
}

\newcommand{\resumeSubSubheading}[2]{
    \begin{tabular*}{0.97\textwidth}{l@{\extracolsep{\fill}}r}
      \textit{\small#1} & \textit{\small #2} \\
    \end{tabular*}\vspace{-5pt}
}

\newcommand{\resumeSubItem}[2]{\resumeItem{#1}{#2}\vspace{-4pt}}

\renewcommand{\labelitemii}{$\circ$}

\newcommand{\resumeSubHeadingListStart}{\begin{itemize}[leftmargin=*]}
\newcommand{\resumeSubHeadingListEnd}{\end{itemize}}
\newcommand{\resumeItemListStart}{\begin{itemize}}
\newcommand{\resumeItemListEnd}{\end{itemize}\vspace{-5pt}}

%-------------------------------------------
%%%%%%  CV STARTS HERE  %%%%%%%%%%%%%%%%%%%%%%%%%%%%

\begin{document}

%----------HEADING-----------------
\begin{tabular*}{\textwidth}{l@{\extracolsep{\fill}}r}
  \textbf{{\Large Liezel Tamon}} & Email : \href{mailto:lizheltamon@gmail.com}{lizheltamon@gmail.com}\\
  %\href{mailto:liezel.tamon@imm.ox.ac.uk}{liezel.tamon@imm.ox.ac.uk}\\
  %\href{mailto:lizheltamon@gmail.com}{lizheltamon@gmail.com}\\
  \href{https://www.imm.ox.ac.uk/people/liezel-tamon}{\textsc{University of Oxford}} & Mobile : +44(0) 7789233360 (WhatsApp) \\
  \href{https://www.linkedin.com/in/liezel-tamon-613348174}{\textsc{LinkedIn}}
  \href{https://github.com/liezeltamon}{\textsc{GitHub}}
\end{tabular*}


%-----------SUMMARY-----------------
%\section{OBJECTIVE}

%A Filipino, final-year PhD in Computational Biology candidate (finishing by April 2023) studying in Oxford, UK, looking for a computational postdoctoral position to apply, hone and acquire skills relevant to data-driven molecular biology research. 

%I am a Filipino, final-year PhD in Computational Biology candidate (finishing by April 2023) studying in Oxford, UK, looking to apply, hone and acquire skills relevant to generating data-driven outputs and solutions. I would like to apply for the part-time data analyst/scientist position but would be open to working full-time after finishing my PhD studies since my residence permit is still valid until August 2023.

%full-time$|$part-time$|$flexitime computational position to apply, hone and acquire skills relevant to Bioinformatics and Computational Biology.  

%A Filipino, final-year PhD in Computational Biology candidate (finishing by April 2023) studying in Oxford, UK, looking for a part-time or flexitime job, where I can apply, hone and acquire statistical, mathematical, and other analytical skills relevant to generating and effectively communicating data-driven solutions or outputs. \textbf{I would like to apply for the Research Assistant position (Grade 5), part-time or flexitime, but would be open to working full-time after finishing my PhD studies since my residence permit is still valid until August 2023.}

%A Filipino, final-year PhD in Computational Biology candidate (finishing by April 2023) studying in Oxford, UK, looking for a postdoctoral position in Computational Biology to apply, hone and acquire skills relevant to data-driven molecular biology research.  

%-----------EDUCATION-----------------
\section{EDUCATION}
  \resumeSubHeadingListStart
    \resumeSubheading
      {University of Oxford}{Oxford, UK}
      {DPhil (PhD) in Medical Sciences (Computational Biology)}{Oct. 2018 -- May 2023}
      \resumeItemListStart
        \resumeItem{Jardine Foundation Postgraduate Scholarship}
        {Full funding}
      \resumeItemListEnd
    \resumeSubheading
      {International Max Planck Research School - University of Goettingen}{Goettingen, DE}
      {Master of Science in Molecular Biology,  1.4 (Excellent)}{Sep. 2016 -- Mar. 2018}
      \resumeItemListStart
        \resumeItem{IMPRS Scholarship}
          {Full funding}
      \resumeItemListEnd
      \resumeSubheading
      {University of the Philippines - Diliman}{Manila, PH}
      {Bachelor of Science in Molecular Biology and Biotechnology, GPA: 3.78 (\textit{Summa cum laude})}{Jun. 2011 -- Jun. 2015}
       \resumeItemListStart
          \resumeItem{Winner, 2015 Sanger Institute Prize}
          
          {International competition open to undergraduates from low- to upper-middle-income countries. The sole winner is granted a fully-funded, 3-month internship at the Wellcome Trust Sanger Institute in Cambridge, UK.}
      \resumeItemListEnd
  \resumeSubHeadingListEnd
  
  
%-----------SKILLS-----------------
\section{SKILLS}
  \resumeSubHeadingListStart
    \resumeSubItem{Programming}
    {Proficient in R programming language, R package development (experience with CRAN deposition), Beginner in Python, Shell scripting, IDEs (RStudio, Jupyter Notebook), {\href{https://github.com/liezeltamon}{Git/GitHub}}, R Markdown, LaTeX, {\href{https://learn.365datascience.com/certificates/CC-1C72E62035/}{SQL}}, routinely use high-performance computing clusters and parallel processing for large-scale data} % \texttt{Rcpp}, Visual Studio Code, Conda
    \resumeSubItem{Statistics}
    {Descriptive, Inferential (Probability and Distributions), Hypothesis testing}
    \resumeSubItem{Predictive analytics}
    {1) Traditional methods (e.g. Regression and Clustering), 2) Machine learning (Implementation using \texttt{caret} R package, experience from published project and thesis, attended in-person ML theory lectures} %, and 3) feature engineering}, currently taking {\href{https://365datascience.com/}{online data science course}})
    \resumeSubItem{Data visualisation}
    {R base, R external libraries (\texttt{ggplot2}, \texttt{Plotly}, \texttt{Shiny} for building interactive web apps and dashboards)}
    \resumeSubItem{Other}
    {Research, Cell and molecular biology experimental methods, Microsoft Office, Poster presentation in scientific conferences} % Bioinformatics
  \resumeSubHeadingListEnd


%-----------EXPERIENCE-----------------
\section{EXPERIENCE}
  \resumeSubHeadingListStart

    \resumeSubheading
      {PhD thesis}{Oxford, UK}
      {Dynamics (Variability) of 3D genome contacts and its implications}{Oct. 2018 -- May 2023}
      \resumeItemListStart
        \resumeItem{Supervised by Dr. Aleksandr B. Sahakyan}{\href{https://www.imm.ox.ac.uk/research/research-groups/sahakyan-group-integrative-computational-biology-and-machine-learning}{Integrative Computational Biology and Machine Learning} - \href{mailto:aleksandr.sahakyan@imm.ox.ac.uk}{aleksandr.sahakyan@imm.ox.ac.uk}}
        \resumeItem{Transition to Computational Biology; Data integration and analyses, statistical analyses, visualisation and interpretation in R}
        {Utilised multiple Hi-C contact datasets from various cell lines and tissues along with other public ``omic" datasets to investigate the core principles governing the 3D genome architecture.}
        \resumeItem{Collaboration}
        {\href{http://www.collepardolab.org/}{Collepardo lab} (Computational Physics of Chromatin) from the University of Cambridge}
     \resumeItemListEnd

     \resumeSubheading
      {R \texttt{Shiny} interactive application}{Oxford, UK}
      {CoreGenomeExplorer: Interactive network visualisation and investigation of the core 3D genome}{}
      \resumeItemListStart
        \resumeItem{R/\texttt{Shiny}}
          {The code is not yet publicly available on GitHub because the app is still being refined and the manuscript is currently in preparation.}
     \resumeItemListEnd

     \resumeSubheading
      {R package \href{https://cran.r-project.org/web/packages/ROptimus/index.html}{\texttt{ROptimus}} (in press, \href{https://www.biorxiv.org/content/10.1101/2022.01.18.476810v1}{bioRxiv preprint})}{Oxford, UK}
      {ROptimus: a general purpose adaptive optimisation engine in R}{}
      \resumeItemListStart
        \resumeItem{Computational tool development}
          {Contributed major algorithm modifications and a tutorial, in charge of revisions and CRAN deposition} % Monte Carlo-based
     \resumeItemListEnd
     
     \resumeSubheading
      {PhD project (\href{https://doi.org/10.1002/anie.202016801}{Published})}{Oxford, UK}
      {\href{https://github.com/SahakyanLab/iMotif_dev}{i-DNA stability predictors}}{2021}
      \resumeItemListStart
        \resumeItem{Modelling: Symbolic regression, \texttt{ROptimus} optimisation, Gradient boosted trees}
          {Used \textsc{Eureqa}, \texttt{ROptimus}, \texttt{caret} and \texttt{xgboost} R packages to implement the 3 aforementioned methods for predicting stability indicators (i.e. melting temperature and pH transition midpoint) of a small subset of i-motif DNA sequences}
          \resumeItem{Collaboration}{Experimental Chemistry groups from Europe and China}
     \resumeItemListEnd

     \resumeSubheading
      {Consulting}{Oxford, UK}
      {Provided a cost-benefit analysis of early and differential diagnosis of the still incurable dementia syndrome}{Jun. 2021}
      \resumeItemListStart
        \resumeItem{\href{https://www.careers.ox.ac.uk/oxford-strategy-challenge}{The Oxford Strategy Challenge (TOSCA)}}
          {Program run by the University of Oxford Careers Service providing a short consultancy training that culminates in 5-man student teams working on real-world projects of companies for 1.5 weeks}
        \resumeItem{Client}
          {\href{https://www.oxfordbraindiagnostics.com/}{Oxford Brain Diagnostics}, providing technology for early and differential dementia diagnosis}
      \resumeItemListEnd

     \newpage
     
     \resumeSubheading
      {MSc thesis (\href{https://doi.org/10.1038/s41418-020-00614-w}{Published})}{Goettingen, DE}
      {Epigenetic regulation of ageing-related changes in bone}{Oct. 2017 - Mar. 2018}
      \resumeItemListStart
        \resumeItem{Supervised by Dr. Zeynab Najafova}{\href{mailto: Zeynab.Najafova@bosch-health-campus.com}{Zeynab.Najafova@bosch-health-campus.com} as part of Prof. Dr. Steven Johnsen's \href{https://johnsenlab.wordpress.com/}{group} - \href{mailto:Steven.Johnsen@bosch-health-campus.com}{Steven.Johnsen@bosch-health-campus.com}}
        \resumeItem{Core molecular biology experimental techniques, Chromatin immunoprecipitation (ChIP) experiment, sequencing library preparation, sequencing data (ChIP-seq and RNA-seq) processing, and data analyses, visualisation and interpretation mainly using \href{https://usegalaxy.org/}{\textsc{Galaxy}}}{Performed H3K27ac and H3K27me3 ChIP on human bone pieces, analysed and correlated ChIP-seq and already available RNA-seq data to characterise the epigenetic changes caused by ageing to bone formation and maintenance}
        \resumeItem{Adopted a newly published protocol}{Helped my MSc supervisor to establish chromatin run-on and sequencing (\href{https://www.nature.com/articles/s41588-018-0244-3#data-availability}{ChRO-seq}) in the laboratory}
     \resumeItemListEnd

     \resumeSubheading
      {MSc rotation projects}{Goettingen, DE}
      {As part of MSc program, 2 months each}{Jan. 2017 - Jun. 2017}
      \resumeItemListStart
        \resumeItem{Characterization of small RNAs derived from human box C/D snoRNA U3}{as part of Dr. Jens Gruber's \href{https://bio-protocol.org/UserHome.aspx?id=1032127}{group}}
        \resumeItem{Influence of epigenetic regulation on vitamin D3 signaling in osteoblasts}{as part of Prof. Dr. Steven Johnsen's \href{https://johnsenlab.wordpress.com/}{group}}
        \resumeItem{Dissecting the roles of microRNAs in stress and Muscular Dystrophy using \textit{Drosophila} as a model}{as part of Prof. Dr. Halyna Shcherbata's \href{https://shcherbatalab.wordpress.com/}{group}}
     \resumeItemListEnd

     \resumeSubheading
      {Sanger Institute Prize Internship}{Cambridge, UK}
      {}{Sep. 2015 - Nov. 2015}
      \resumeItemListStart
        \resumeItem{Supervised by Dr. Chi Wong}{\href{mailto: chi.wong@sanger.ac.uk}{chi.wong@sanger.ac.uk} as part of Dr. David Adams's \href{https://www.sanger.ac.uk/group/adams-group/}{Experimental Cancer Genetics group} - \href{mailto:david.adams@sanger.ac.uk}{david.adams@sanger.ac.uk}}
        \resumeItem{Molecular biology experimental techniques}
          {Molecular cloning, tested CRISPR/Cas9 efficiency through surveyor assay, generated conditional overexpression construct for mouse as part of characterising the \href{https://www.nature.com/articles/ng.2947}{POT1 variants} linked to familial melanoma predisposition)}
     \resumeItemListEnd
     
     \resumeSubheading
      {BSc thesis}{Manila, PH}{\href{https://www.researchgate.net/publication/343770282_Co-Regulation_of_KRAS_and_its_Putative_ceRNAs_KRASP1_ZNF148_and_ONECUT2_in_Colorectal_Cancer}{Co-regulation of KRAS and ONECUT2 Oncogenes by Shared MicroRNAs in cancer context}}{Apr. 2014 - Jun. 2015}
      \resumeItemListStart
        \resumeItem{Supervised by Prof. Dr. Reynaldo L. Garcia (Cantab)}{\href{https://www.dmbel-nimbb.com/home}{Disease Molecular Biology and Epigenetics Laboratory}; \href{mailto:reygarcia@mbb.upd.edu.ph}{reygarcia@mbb.upd.edu.ph}}
        \resumeItem{Molecular biology experimental techniques}
          {Performed molecular cloning, cell culture, semi-quantitative PCR, dual luciferase assay, etc. to study the regulatory effects of the potential competition for hsa- miR-181a-5p and hsa-miR-15a-3p between KRAS and its \textit{in silico} competing endogenous RNA (ceRNA), ONECUT2, in the context of colorectal cancer}
     \resumeItemListEnd

  \resumeSubHeadingListEnd


%-----------PUBLICATIONS-----------------
\section{PUBLICATION}
  \resumeSubHeadingListStart
    \resumeSubItem{\href{https://scholar.google.com/citations?user=XYWobTUAAAAJ&hl=en}{Google Scholar}}{https://scholar.google.com/citations?user=XYWobTUAAAAJ\&hl=en}
    \resumeSubItem{\href{https://github.com/SahakyanLab/ROptimus}{\texttt{ROptimus} R package}}
    {\href{https://www.biorxiv.org/content/10.1101/2022.01.18.476810v1}{bioRxiv preprint}, in press}
    \resumeSubItem{\href{https://doi.org/10.1002/anie.202016801}{Angewandte Chemie 2021}}
    {Contributed the i-DNA stability predictors}
    \resumeSubItem{\href{https://doi.org/10.1038/s41418-020-00614-w}{Nature Cell Death Differ. 2020}}
    {Contributed results from my MSc thesis}
    \resumeSubItem{\href{https://doi.org/10.21769/BioProtoc.3000}{Bio-Protocol 2018}}
    {Contributed experiments during MSc rotation}
  \resumeSubHeadingListEnd

%-----------MEMBERSHIPS-----------------
\section{MEMBERSHIPS}

    \resumeSubheading
      {\href{https://oxfordphilippinessociety.web.ox.ac.uk/home}{Oxford Philippines Society}}{Oxford, UK}
      {}{Oct. 2018 - present}
      \resumeItemListStart
        \resumeItem{Treasurer (Oct. 2018 - Sep. 2021) \& Member (present)}{Aside from managing finances and memberships, I organised monthly talks and social events as well.}
     \resumeItemListEnd
     
    \resumeSubheading
      {\href{https://ouvc.notion.site/ouvc/OUVC-Homepage-1d969e57c26c4426afc33d509b8736ca}{Oxford University Volleyball Club}}{Oxford, UK}
      {}{Oct. 2018 - Dec. 2019, Oct. 2022 - present}
      \resumeItemListStart
        \resumeItem{Social media officer (Oct. 2019 - Dec. 2019) \& Member - Women's 2nd team (present)}{I love playing volleyball and other racquet sports like tennis, squash and table tennis!}
     \resumeItemListEnd
     
  \resumeSubheading
      {\href{https://psysc.org/}{Philippine Society of Youth Science Clubs, Inc.}}{Manila, PH}
      {}{Oct. 2012 - Jun. 2015}
      \resumeItemListStart
        \resumeItem{Events organiser \& member}{PSYSC promotes the public understanding of science, technology and the environment. I worked in as well as spearheaded committees organising national science camps, events and competitions attended by thousands of secondary school students.}
     \resumeItemListEnd

     %\resumeSubheading
     % {\href{https://en-gb.facebook.com/AngKlubTala/}{Klub Tala}}{Manila, PH}
     % {}{Oct. 2012 - Jun. 2015}
     % \resumeItemListStart
     %   \resumeItem{Vice president for Catechism (2013 - 2014) \& Member}{Klub Tala promotes the holistic development of Filipino children. As VP, I prepared content and materials for our weekly visits to an underprivileged community near our University and gave academic tutorials, catechism and values classes, and workshops (e.g. fashion, art) mixed in with fun activities and games to children.}

     %\resumeItemListEnd

%-----------OTHERS-----------------
\section{OTHERS}
  \resumeSubHeadingListStart
    \resumeSubItem{Ed Padlan Awardee, 2nd place}{Sep. 2014, Manila PH; Given to the Top 2 (out of 33) students in my 3rd year in University.}
    \resumeSubItem{Champion, Philippine National Chemistry Olympiad}{Apr. 2010, Manila PH}
  \resumeSubHeadingListEnd
  
%-------------------------------------------

\end{document}
