%-------------------------
% Resume in Latex
% Author : Sourabh Bajaj
% License : MIT
%------------------------

\documentclass[legalpaper,11pt]{article}

\usepackage{latexsym}
\usepackage[empty]{fullpage}
\usepackage{titlesec}
\usepackage{marvosym}
\usepackage[usenames,dvipsnames]{color}
\usepackage{verbatim}
\usepackage{enumitem}
%\usepackage[hidelinks]{hyperref}
\usepackage{hyperref}
\hypersetup{
    colorlinks=true,
    linkcolor=blue,
    filecolor=magenta,      
    urlcolor=blue%,
    %pdfpagemode=FullScreen
}

\usepackage{fancyhdr}
\usepackage[english]{babel}
\usepackage{tabularx}

\pagestyle{fancy}
\fancyhf{} % clear all header and footer fields
\fancyfoot{}
\renewcommand{\headrulewidth}{0pt}
\renewcommand{\footrulewidth}{0pt}

% Adjust margins
\addtolength{\oddsidemargin}{-0.5in}
\addtolength{\evensidemargin}{-0.5in}
\addtolength{\textwidth}{1in}
\addtolength{\topmargin}{-.5in}
\addtolength{\textheight}{1.0in}

\urlstyle{same}

\raggedbottom
\raggedright
\setlength{\tabcolsep}{0in}

% Sections formatting
\titleformat{\section}{
  \vspace{-4pt}\scshape\raggedright\large
}{}{0em}{}[\color{black}\titlerule \vspace{-5pt}]

%-------------------------
% Custom commands
\newcommand{\resumeItem}[2]{
  \item\small{
    \textbf{#1}{: #2 \vspace{-2pt}}
  }
}

\newcommand{\resumeSubheading}[4]{
  \vspace{-1pt}\item
    \begin{tabular*}{0.97\textwidth}[t]{l@{\extracolsep{\fill}}r}
      \textbf{#1} & #2 \\
      \textit{\small#3} & \textit{\small #4} \\
    \end{tabular*}\vspace{-5pt}
}

\newcommand{\resumeSubSubheading}[2]{
    \begin{tabular*}{0.97\textwidth}{l@{\extracolsep{\fill}}r}
      \textit{\small#1} & \textit{\small #2} \\
    \end{tabular*}\vspace{-5pt}
}

\newcommand{\resumeSubItem}[2]{\resumeItem{#1}{#2}\vspace{-4pt}}

\renewcommand{\labelitemii}{$\circ$}

\newcommand{\resumeSubHeadingListStart}{\begin{itemize}[leftmargin=*]}
\newcommand{\resumeSubHeadingListEnd}{\end{itemize}}
\newcommand{\resumeItemListStart}{\begin{itemize}}
\newcommand{\resumeItemListEnd}{\end{itemize}\vspace{-5pt}}

%-------------------------------------------
%%%%%%  CV STARTS HERE  %%%%%%%%%%%%%%%%%%%%%%%%%%%%

\begin{document}

%----------HEADING-----------------
\begin{tabular*}{\textwidth}{l@{\extracolsep{\fill}}r}
  \textbf{{\Large Liezel Uy Tamon}} & Email : \href{mailto:liezel.tamon@imm.ox.ac.uk}{liezel.tamon@imm.ox.ac.uk} \& \href{mailto:lizheltamon@gmail.com}{lizheltamon@gmail.com}\\
  %\href{mailto:liezel.tamon@imm.ox.ac.uk}{liezel.tamon@imm.ox.ac.uk}\\
  %\href{mailto:lizheltamon@gmail.com}{lizheltamon@gmail.com}\\
  %\href{https://www.imm.ox.ac.uk/people/liezel-tamon}{\textsc{University of Oxford}} & Mobile : +44(0) 7789233360 (WhatsApp) \\
  %\href{https://www.imm.ox.ac.uk/people/liezel-tamon}{\textsc{}} & Mobile : +44(0) 7789233360 (WhatsApp) \\
  University of Oxford & Mobile : +44(0) 7789233360 \\
  \href{https://www.linkedin.com/in/liezel-tamon-613348174}{\textsc{LinkedIn}}
  \href{https://github.com/liezeltamon}{\textsc{GitHub}}
\end{tabular*}


%-----------SUMMARY-----------------
%\section{OBJECTIVE}

%A Filipino, final-year PhD in Computational Biology candidate (finishing by April 2023) studying in Oxford, UK, looking for a computational postdoctoral position to apply, hone and acquire skills relevant to data-driven molecular biology research. 

%I am a Filipino, final-year PhD in Computational Biology candidate (finishing by April 2023) studying in Oxford, UK, looking to apply, hone and acquire skills relevant to generating data-driven outputs and solutions. I would like to apply for the part-time data analyst/scientist position but would be open to working full-time after finishing my PhD studies since my residence permit is still valid until August 2023.

%full-time$|$part-time$|$flexitime computational position to apply, hone and acquire skills relevant to Bioinformatics and Computational Biology.  

%A Filipino, final-year PhD in Computational Biology candidate (finishing by April 2023) studying in Oxford, UK, looking for a part-time or flexitime job, where I can apply, hone and acquire statistical, mathematical, and other analytical skills relevant to generating and effectively communicating data-driven solutions or outputs. \textbf{I would like to apply for the Research Assistant position (Grade 5), part-time or flexitime, but would be open to working full-time after finishing my PhD studies since my residence permit is still valid until August 2023.}

%A Filipino, final-year PhD in Computational Biology candidate (finishing by April 2023) studying in Oxford, UK, looking for a postdoctoral position in Computational Biology to apply, hone and acquire skills relevant to data-driven molecular biology research.  

%-----------EDUCATION-----------------
\section{EDUCATION}
  \resumeSubHeadingListStart
    \resumeSubheading
      {University of Oxford}{Oxford, UK}
      {DPhil (PhD) in Medical Sciences - Computational Biology}{Oct. 2018 -- Jun. 2023}
      \resumeItemListStart
        \resumeItem{\href{https://www.jardines.com/en/sustainability/our-strategy/shaping-social-inclusion/jardine-foundation?tab=scholarship-schemes}{Jardine Foundation Postgraduate Scholarship}}
        {Full funding}
      \resumeItemListEnd
    \resumeSubheading
      {\href{https://uni-goettingen.de/en/663690.html}{International Max Planck Research School - University of Goettingen}}{Goettingen, DE}
      {Master of Science in Molecular Biology,  1.4 (Excellent)}{Sep. 2016 -- Mar. 2018}
      \resumeItemListStart
        \resumeItem{IMPRS Scholarship}
          {Full funding}
      \resumeItemListEnd
      \resumeSubheading
      {University of the Philippines - Diliman}{Manila, PH}
      {Bachelor of Science in Molecular Biology and Biotechnology, GPA: 3.78 (\textit{Summa cum laude})}{Jun. 2011 -- Jun. 2015}
       \resumeItemListStart
          \resumeItem{Winner, 2015 \href{https://www.sanger.ac.uk/about/study/the-sanger-prize/}{Sanger Institute Prize}}
          
          {International competition open to undergraduates from low- to upper-middle-income countries. The sole winner is granted a fully-funded, 3-month internship at the Wellcome Trust Sanger Institute in Cambridge, UK.}
      \resumeItemListEnd
  \resumeSubHeadingListEnd
  
  
%-----------SKILLS-----------------
\section{RELEVANT SKILLS}
  \resumeSubHeadingListStart
    \resumeSubItem{Programming}
    {R programming language, R package development (experience with \href{https://cran.r-project.org/web/packages/ROptimus/index.html}{CRAN} package deposition), Linux command line, Python (beginner), Conda, Integrated development environment / IDEs (RStudio, Jupyter Notebook), Routinely using high-performance computing clusters and creating parallel processing workflows to deal with large-scale data, {\href{https://learn.365datascience.com/certificates/CC-1C72E62035/}{SQL}} (learned through an online course)} % \texttt{Rcpp}, Visual Studio Code
    \resumeSubItem{Data visualisation}
    {R base, R external libraries (e.g. \texttt{ggplot2}, \texttt{Shiny})}
    \resumeSubItem{Documentation and reproducibility}{{\href{https://github.com/liezeltamon}{Git/GitHub}}, Conda and renv for reproducible environments, Quarto and R Markdown, Overleaf, Microsoft Office}
    \resumeSubItem{Predictive analytics}
    {1) Traditional methods (e.g. Regression and Clustering), 2) Machine learning (Applied using \texttt{caret} R package for a published project, experience from thesis work, attended in-person ML theory lectures} %, and 3) feature engineering}, currently taking {\href{https://365datascience.com/}{online data science course}})
    \resumeSubItem{Statistics}
    {Descriptive, Inferential, Hypothesis testing}
    \resumeSubItem{Other}
    {Cell and molecular biology experimental methods}
  \resumeSubHeadingListEnd

%-----------CURRENT ROLES-----------------
\section{CURRENT ROLES}

I manage simultaneous responsibilities, including teaching (in-person) computational biology courses on a termly basis (20 to 30 days per academic term) and working on purely computational projects as well as doing computational analyses of newly generated data in collaboration with experimental groups.

  \resumeSubHeadingListStart

    \resumeSubheading
      {Postdoctoral computational biologist}{Oxford, UK}
      {}{Jun. 2023 -- present}
      \resumeItemListStart
        \resumeItem{Supervised by Assoc. Prof. David Sims}{\href{https://www.rdm.ox.ac.uk/people/david-sims}{Transcriptional regulation in neuroscience}}
        \resumeItem{Single-cell data analyses, collaboration}{Currently working on analysing single-cell transcriptomic datasets from two independent projects involving: 1) Genetically modified human brain organoids in collaboration with a group in Germany and 2) human-induced pluripotent stem cell (iPSC)-derived models for studying Parkinson's disease in collaboration with a group in Oxford}
     \resumeItemListEnd

     \resumeSubheading
      {Computational Biology instructor}{Oxford, UK}
      {}{Jun. 2023 -- present}
      \resumeItemListStart
        \resumeItem{{\href{https://www.imm.ox.ac.uk/research/units-and-centres/mrc-wimm-centre-for-computational-biology/training/oxford-biomedical-data-science-training-programme}{Oxford Biomedical Data Science Training Programme}}}{Led by Assoc. Prof. David Sims; OBDS aims to enable biomedical researchers to apply data science and computational biology methods to their own genomic data}
        \resumeItem{Teaching}{I am one of the instructors for these modules i.e. Genomics on the Linux command line, R for data science and genomics, and Introduction to single-cell RNAseq data analysis using R. To further develop my skills, I recently completed the Python programming, Data science and Single-cell RNAseq using Python modules. Course materials on {\href{https://github.com/kevinrue/OBDS_slides}{GitHub}}.}
     \resumeItemListEnd

  \resumeSubHeadingListEnd
  
%-----------EXPERIENCE-----------------
% Template - Supervisor and group, gist of project / context, skills acquired
\section{RESEARCH EXPERIENCE}
  \resumeSubHeadingListStart
  
     \resumeSubheading
      {PhD thesis (Manuscript submitted)}{Oxford, UK}
      {Dynamics (Variability) of 3D genome contacts and its implications}{Oct. 2018 -- Jun. 2023}
      \resumeItemListStart
        \resumeItem{Supervised by Dr. Aleksandr B. Sahakyan}{\href{https://www.imm.ox.ac.uk/research/research-groups/sahakyan-group-integrative-computational-biology-and-machine-learning}{Integrative Computational Biology and Machine Learning} - \href{mailto:aleksandr.sahakyan@imm.ox.ac.uk}{aleksandr.sahakyan@imm.ox.ac.uk}}
        \resumeItem{Transition to Computational Biology; Data integration and analyses, statistical analyses, visualisation and interpretation in R}
        {Leveraged multiple Hi-C contact datasets sourced from various cell lines and tissues, in conjunction with other publicly available "omic" datasets, to delve into the core principles underlying 3D genome architecture, with a focus on understanding the contribution of sequence-based factors (e.g. k-meric composition, DNA secondary structure, repeat elements, motifs) to genomic contact formation}
        \resumeItem{Collaboration}
        {\href{http://www.collepardolab.org/}{Collepardo lab} (Computational Physics of Chromatin) from the University of Cambridge}
        \resumeItem{Manuscript}{Please feel free to contact me if a copy is needed to better understand the project and the skills acquired}
     \resumeItemListEnd
     
     \resumeSubheading
      {R \texttt{Shiny} interactive application}{Oxford, UK}
      {CoreGenomeExplorer: Interactive network visualisation and investigation of the core 3D genome}{}
      \resumeItemListStart
        \resumeItem{R/\texttt{Shiny}}
          {Developed the app to facilitate interactive exploration and analysis of processed contact datasets primarily utilised for the thesis project. The code is not yet publicly available on GitHub because the app is still being refined and the manuscript is currently in preparation.}
     \resumeItemListEnd

     \resumeSubheading
      {R package \href{https://cran.r-project.org/web/packages/ROptimus/index.html}{\texttt{ROptimus}} (\href{https://doi.org/10.1093/bioinformatics/btad292}{published})}{Oxford, UK}
      {ROptimus: a general purpose adaptive optimisation engine in R}{}
      \resumeItemListStart
        \resumeItem{Computational tool development}
          {Contributed major algorithm modifications and a tutorial, also in charge of journal revisions and CRAN deposition} % Monte Carlo-based
        \resumeItem{Application}{Due to ROptimus' versatility in tackling various optimisation problems, I have used it for two distinct purposes, 1) generating control sets of contacts for my thesis and 2) developing a predictor for i-motif DNA stability}
     \resumeItemListEnd
     
     \resumeSubheading
      {PhD project (\href{https://doi.org/10.1002/anie.202016801}{Published}, \href{https://github.com/SahakyanLab/iMotif_dev}{GitHub})}{Oxford, UK}
      {i-motif DNA stability predictors}{2021}
      \resumeItemListStart
        \resumeItem{Modelling: Symbolic regression, \texttt{ROptimus} optimisation, Gradient boosted trees}
          {Used \textsc{Eureqa}, \texttt{ROptimus}, \texttt{caret} and \texttt{xgboost} R packages to implement the 3 aforementioned methods for predicting stability indicators (i.e. melting temperature and pH transition midpoint) of a small subset of i-motif DNA sequences}
          \resumeItem{Collaboration}{Experimental Chemistry groups from Europe and China}
     \resumeItemListEnd

     \resumeSubheading
      {Brief consulting experience}{Oxford, UK}
      {Provided a cost-benefit analysis of early and differential diagnosis of the still incurable dementia syndrome}{Jun. 2021}
      \resumeItemListStart
        \resumeItem{\href{https://www.careers.ox.ac.uk/oxford-strategy-challenge}{The Oxford Strategy Challenge (TOSCA)}}
          {Program run by the University of Oxford Careers Service providing a short consultancy training that culminates in 5-man student teams working on real-world projects of companies for 1.5 weeks}
        \resumeItem{Client}
          {\href{https://www.oxfordbraindiagnostics.com/}{Oxford Brain Diagnostics}, providing technology for early and differential dementia diagnosis}
      \resumeItemListEnd
     
     \resumeSubheading
      {MSc thesis}{Goettingen, DE}
      {Epigenetic regulation of ageing-related changes in bone}{Oct. 2017 - Mar. 2018}
      \resumeItemListStart
        \resumeItem{Supervised by Dr. Zeynab Najafova}{\href{mailto: Zeynab.Najafova@bosch-health-campus.com}{Zeynab.Najafova@bosch-health-campus.com} as part of Prof. Dr. Steven Johnsen's \href{https://johnsenlab.wordpress.com/}{group} - \href{mailto:Steven.Johnsen@bosch-health-campus.com}{Steven.Johnsen@bosch-health-campus.com}}
        \resumeItem{Core molecular biology experimental techniques, Chromatin immunoprecipitation (ChIP) experiment, sequencing library preparation; sequencing data (ChIP-seq and RNA-seq) processing, analyses and visualisation mainly using \href{https://usegalaxy.org/}{\textsc{Galaxy}}}{Performed H3K27ac and H3K27me3 ChIP on human bone pieces, analysed and correlated ChIP-seq and already available RNA-seq data to characterise the epigenetic changes caused by ageing to bone formation and maintenance}
        \resumeItem{Adopted a newly published protocol}{Helped my MSc supervisor to establish chromatin run-on and sequencing (\href{https://www.nature.com/articles/s41588-018-0244-3#data-availability}{ChRO-seq}) in the laboratory, generated ChRO-seq data reported in \href{http://dx.doi.org/10.1136/gutjnl-2022-328154}{BMJ Journals Gut 2023}}
     \resumeItemListEnd

     \resumeSubheading
      {MSc rotation projects}{Goettingen, DE}
      {As part of MSc program, 2 months each}{Jan. 2017 - Jun. 2017}
      \resumeItemListStart
        \resumeItem{Characterization of small RNAs derived from human box C/D snoRNA U3}{as part of Dr. Jens Gruber's \href{https://bio-protocol.org/UserHome.aspx?id=1032127}{group}}
        \resumeItem{Influence of epigenetic regulation on vitamin D3 signaling in osteoblasts}{as part of Prof. Dr. Steven Johnsen's \href{https://johnsenlab.wordpress.com/}{group}}
        \resumeItem{Dissecting the roles of microRNAs in stress and Muscular Dystrophy using \textit{Drosophila} as a model}{as part of Prof. Dr. Halyna Shcherbata's \href{https://shcherbatalab.wordpress.com/}{group}}
     \resumeItemListEnd

     \resumeSubheading
      {Sanger Institute Prize internship}{Cambridge, UK}
      {}{Sep. 2015 - Nov. 2015}
      \resumeItemListStart
        \resumeItem{Supervised by Dr. Chi Wong}{\href{mailto: chi.wong@sanger.ac.uk}{chi.wong@sanger.ac.uk} as part of Dr. David Adams's \href{https://www.sanger.ac.uk/group/adams-group/}{Experimental Cancer Genetics group} - \href{mailto:david.adams@sanger.ac.uk}{david.adams@sanger.ac.uk}}
        \resumeItem{Molecular biology experimental techniques}
          {Molecular cloning, tested CRISPR/Cas9 efficiency through surveyor assay, generated conditional overexpression construct for mouse as part of characterising the \href{https://www.nature.com/articles/ng.2947}{POT1 variants} linked to familial melanoma predisposition)}
     \resumeItemListEnd
     
     \resumeSubheading
      {BSc thesis}{Manila, PH}{\href{https://www.researchgate.net/publication/343770282_Co-Regulation_of_KRAS_and_its_Putative_ceRNAs_KRASP1_ZNF148_and_ONECUT2_in_Colorectal_Cancer}{Co-regulation of KRAS and ONECUT2 Oncogenes by Shared MicroRNAs in cancer context}}{Apr. 2014 - Jun. 2015}
      \resumeItemListStart
        \resumeItem{Supervised by Prof. Dr. Reynaldo L. Garcia (Cantab)}{\href{https://www.dmbel-nimbb.com/home}{Disease Molecular Biology and Epigenetics Laboratory}; \href{mailto:reygarcia@mbb.upd.edu.ph}{reygarcia@mbb.upd.edu.ph}}
        \resumeItem{Molecular biology experimental techniques}
          {Performed molecular cloning, cell culture, semi-quantitative PCR, dual luciferase assay, etc. to study the regulatory effects of the potential competition for hsa- miR-181a-5p and hsa-miR-15a-3p between KRAS and its \textit{in silico} competing endogenous RNA (ceRNA), ONECUT2, in the context of colorectal cancer}
     \resumeItemListEnd

  \resumeSubHeadingListEnd


%-----------PUBLICATIONS-----------------
\section{PUBLICATION}
  \resumeSubHeadingListStart
    \resumeSubItem{\href{https://scholar.google.com/citations?user=XYWobTUAAAAJ&hl=en}{Google Scholar}}{https://scholar.google.com/citations?user=XYWobTUAAAAJ\&hl=en}
    \resumeSubItem{Johnson NAG*, \underline{Tamon L}*, Liu X, Sahakyan AB \texttt{ROptimus} \href{https://doi.org/10.1093/bioinformatics/btad292}{Bioinformatics 2023},
    \href{https://github.com/SahakyanLab/ROptimus}{GitHub}}
     {Major algorithm modifications, publication, CRAN deposition}
    \resumeSubItem{Hamdan FH, Abdelrahman AM, Kutschat AP, Wang X, Ekstrom TL, Jalan-Sakrikar N, Wegner Wippel C, Taheri N, \underline{Tamon L}...Najafova Z, Hessmann E, Truty MJ, Johnsen SA \href{http://dx.doi.org/10.1136/gutjnl-2022-328154}{BMJ Journals Gut 2023}}
    {Contributed to generation of ChRO-seq data}
    \resumeSubItem{Cheng M, Qiu D, \underline{Tamon L}...Sahakyan AB, Zhou J, Mergny JL \href{https://doi.org/10.1002/anie.202016801}{Angew Chem Int Ed Engl. 2021}}
    {Contributed the i-motif DNA stability predictors}
    \resumeSubItem{Najafova Z, Liu P, Wegwitz F, Ahmad M, \underline{Tamon L}...Johnsen SA, Tuckermann J \href{https://doi.org/10.1038/s41418-020-00614-w}{Nature Cell Death Differ. 2021}}
    {Contributed to experiments as part of my MSc thesis}
    \resumeSubItem{Lemus-Diaz N, \underline{Tamon L}, Gruber J. \href{https://doi.org/10.21769/BioProtoc.3000}{Bio-Protocol 2018}}
    {Contributed to experiments as part of my MSc rotation}
  \resumeSubHeadingListEnd

\section{ADDITIONAL INFORMATION}
  \resumeSubHeadingListStart
    \resumeSubItem{Right to work in the UK}{I am a Filipino citizen with right to full-time work under a \href{https://www.gov.uk/graduate-visa}{Graduate Visa} valid until 31 July 2026}
    \resumeSubItem{Interests}
    {I love playing volleyball! I regularly play socially and also train as a member of the \href{http://www.oxfordvolleyball.co.uk}{Oxford Volleyball Club} Oxford Falcons team to participate in tournaments. I also like playing racquet sports like tennis, squash and table tennis.}
  \resumeSubHeadingListEnd

\end{document}
